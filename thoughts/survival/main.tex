% \documentclass{IEEEtran}
\documentclass[12pt]{article}
\usepackage[utf8]{inputenc}

\usepackage[margin=1in]{geometry}

\usepackage{librebaskerville}
\usepackage[T1]{fontenc}

\usepackage{setspace}
\doublespacing

\usepackage{url}
\def\UrlBreaks{\do\/\do-}

\usepackage{graphicx}
\usepackage[labelformat=empty]{caption}
\usepackage{subcaption}
\usepackage{notoccite}

% \usepackage{xcolor}
% \pagecolor[rgb]{0.1,0.1,0.1}
% \color[rgb]{0.7,0.7,0.7}

\newenvironment{quote_1in}%
  {\list{}{\leftmargin=1in\rightmargin=1in}\item[]}%
  {\endlist}

\interfootnotelinepenalty=10000

\setlength{\parskip}{\baselineskip}
\setlength{\parindent}{0pt}

\widowpenalty10000
\clubpenalty10000

\title{Survival \\[-6pt]}
\author{\vspace{-\baselineskip}Peter Lillian}
\date{\vspace{-\baselineskip}August 2021}

\begin{document}

\frenchspacing

\maketitle


% \begin{quote_1in}
%     \textit{When Marduk commanded me to give justice to the people of the land and to let them have good governance, I set forth... to make justice appear in the land, to destroy the evil and the wicked, that the strong might not oppress the weak.}
    
%     ---Hammurabi of Babylonia \cite{driver2007babylonian}
% \end{quote_1in}

The poor tiny snail must have been born there, in our trash can. I knew it would die there, too, if I didn’t do anything. And so the next ten minutes of my life were spent coaxing it onto the back of some spam mail I found lying on the counter. After walking outside carefully—he had refused to stick onto the paper and his shell was sliding around the words “URGENT OPEN NOW”—I made my way across the lawn. Poor snail. Just as I was trying to shake it off, it finally stuck onto the paper. As ungrateful as it was, eventually its sloppy trail led off the envelope, attracted by another snail that it seemed to like very much. Thus the snail was saved from a short life in the garbage.

I figured I’d leave them to their business.

I turned around heroically to go back inside, but only a few steps from the door my foot was met by a faint crunch. I could feel the snail guts dripping off my bare toes.

It’s just how morbidly poetic the whole damn thing was. I save a snail only to kill another. But we aren’t that different from them. Made up of cells containing simpler cells, multitudes of molecules made up of atoms upon atoms. We look for food and shelter, just as the snail did in the trash. We seek love, just as he did on the lawn. The snail wants survival and reproduction, same as all living beings.

But we’re so different from the snail that it couldn’t dream of recognizing us as even alive. Its senses are limited to what would be useful in its snail world. I could never explain to the snail what I had done, or even what a trash can was.

So am I a god to this snail? Possibly. But if we’re in turn minuscule next to the sheer weight of the universe, how can we say we’re the top of the pyramid? Is there a limit to complexity? We can easily imagine a being slightly less complex, less intelligent than ourselves. And just as easily one slightly more so. Clearly that being could imagine an even greater being. As technology improves, it seems only a matter of time until these beings exist and begin climbing the pyramid higher and higher.

That’s a story for another time though. Is there a cap to the pyramid? Even if the universe is infinite, in about 10\textsuperscript{100} years the last black hole will fizzle out of existence, and the cosmos will become cold and dark for eternity \cite{frautschi1982entropy}. Eventually there will be no usable energy or mass—the concept of time will become meaningless in this sea of maximum entropy.

But before this happens, our civilization will continue to advance. Our descendants will become more intelligent at an increasing rate as we spread throughout the galaxies. It’s likely that by harvesting energy from black holes \cite{penrose1971extraction, press1972floating}, we’ll be able to persist for the 10\textsuperscript{100} years until they fizzle away. And during this unimaginable timespan, there will eventually emerge beings that \textit{we} would consider gods. Life on Earth has only existed for several billion years \cite{mojzsis1996evidence}—imagine how far life could progress in a billion billion billion years. Now add 30 more billions.

Beings that much more advanced than us would certainly see us as we see the snail. We wouldn’t be able to comprehend what they were any more than the snail can comprehend us.

But we do know one thing about them. The supreme task for these beings will be to find a way of escaping the heat death of the universe. Somehow it may be possible to reverse the march of entropy or even find a way to leave this universe.

So it’s funny—these gods will use their powers to search for energy, a new universe, a way to keep living. Food, shelter, reproduction. Just like the snail.


\bibliographystyle{IEEEtran}
\bibliography{biblio}

\end{document}
